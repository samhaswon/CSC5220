\documentclass[letterpaper]{article}
\usepackage[utf8]{inputenc}
\usepackage{url}
\usepackage{aaai}
\usepackage{times}
\usepackage{helvet}
\usepackage{courier}

\begin{filecontents}{references.bib}
@misc{nrel_routee,
    author = "National Renewable Energy Laboratory NREL",
    organization = "National Renewable Energy Laboratory",
    year = "2017",
    title = "Route{E}: Route {E}nergy {P}rediction {M}odel",
    url = "https://www.nrel.gov/transportation/route-energy-prediction-model.html"
}

@misc{google_2023_environmental_report,
    author = "Google",
    organization = "Google Inc.",
    year = "2023",
    title = "Google 2023 {E}nvironmental {R}eport",
    url = "https://www.gstatic.com/gumdrop/sustainability/google-2023-environmental-report.pdf"
}

\end{filecontents}

\frenchspacing
\setlength{\pdfpagewidth}{8.5in}
\setlength{\pdfpageheight}{11in}
\pdfinfo{
/Title (Temperature-Dependent Estimation of Vehicle Fuel Economy)
/Author (Sindu Chitraju, Samuel Howard, Seif Ilkbarieh, Om Solanki, Andrew Wheeler)}
\setcounter{secnumdepth}{0}  
 \begin{document}

% The file aaai.sty is the style file for AAAI Press 
% proceedings, working notes, and technical reports.

% Remove the copyright from the footer
\nocopyright

\title{Temperature-Dependent Estimation of Vehicle Fuel Economy}
\author{Sindu Chitraju, Samuel Howard, Seif Ilkbarieh, Om Solanki, Andrew Wheeler\\
Tennessee Technological University
}

\maketitle

\begin{abstract}
\begin{quote}
    This project proposes to investigate the application of Recurrent Neural Networks (RNNs) to predict vehicle fuel economy using readily available on-board diagnostics II (OBD-II) data. 
    Leveraging the temporal dependencies inherent in driving patterns, we aim to develop a model that accurately predicts fuel consumption based on a sequence of sensor readings. 
    This approach offers a personalized and adaptable alternative to generalized fuel economy estimates, further contextualized with external climate data for even more accurate estimates.
    We will explore various RNN architectures and evaluate their performance using real-world driving data collected from a personal vehicle. 
    This research has the potential to contribute to fuel efficiency optimization strategies, personalized driver feedback systems, and more accurate route planning tools, building upon existing work like the Route Energy Prediction Model (REPM) used in Google Maps. 
\end{quote}
\end{abstract}

\section{Introduction}

\noindent Fuel economy is a critical concern with its economic and environmental implications. 
While standardized tests provide a general idea of the performance of a vehicle, real-world fuel consumption varies significantly based on driving style, traffic conditions, and other factors.
This project addresses the need for route-specific, weather dependent fuel economy prediction by leveraging the power of RNNs to model the complex temporal relationships within driving data.
We hypothesize that RNNs can learn the intricate dependencies between many of the variables that contribute to fuel economy including temperature, a metric that is not often accounted for.
The result of this work could empower drivers to optimize their driving habits whilst vehicle manufacturers could optimize aspects of vehicle drivetrains to account for environmental conditions.

\section{Related Work}

This project builds upon existing research in fuel economy modeling and route planning. 
The NREL's REPM, which powers fuel efficient routing in Google Maps, 
demonstrates the value of accurate fuel consumption prediction, 
preventing more than 1.2 million metric tons of carbon emissions \cite{google_2023_environmental_report}.
While REPM uses aggregated and generalized data from a few vehicles, 
our approach leverages the richness and breadth of vehicle sensor data from a single vehicle for better accuracy \cite{nrel_routee}.



\section{Expected Results}

We expect that the model will outperform the RouteE model with real-world observed vehicular data. 
As our model will be working with environmental and route data, 
it will better be able to account for the conditions as well as the performance characteristics of the tested vehicle.
Furthermore, real-time inferencing using our model would allow a driver to glean yet greater fuel economy while driving.
Lastly, our model could enable a driver to save fuel by planning their trip ahead of time to optimize the time they leave for fuel economy.

\section{Conclusion and Future Work}

This project proposes to investigate the application of Recurrent Neural Networks (RNNs) to predict vehicle fuel economy using readily available on-board diagnostics II (OBD-II) data. 
By leveraging the temporal dependencies inherent in driving patterns, 
we aim to develop a personalized and adaptable model that accurately predicts fuel consumption based on a sequence of sensor readings. 
This approach offers a significant advantage over generalized fuel economy estimates, 
potentially empowering drivers to optimize their driving habits for improved fuel efficiency. 
Furthermore, we will investigate the potential of incorporating external climate data to enhance prediction accuracy over contemporary models.

The successful development of such a model has broad implications. 
It could contribute to personalized driver feedback systems, 
enabling real-time fuel efficiency coaching. 
Moreover, the predicted fuel economy could be integrated into route planning tools, 
complementing existing solutions like the Route Energy Prediction Model (REPM) used in Google Maps, 
by providing personalized and dynamic fuel consumption estimates whilest en route. 
This research aligns with the broader goals of promoting fuel conservation and sustainable transportation.

\bibliographystyle{aaai} 
\bibliography{references.bib}


\end{document}