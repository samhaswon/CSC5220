\documentclass[letterpaper]{article}
\usepackage[utf8]{inputenc}
\usepackage{url}
\usepackage{aaai}
\usepackage{times}
\usepackage{helvet}
\usepackage{courier}

\begin{filecontents}{references.bib}
@misc{nrel_routee,
    author = "National Renewable Energy Laboratory NREL",
    organization = "National Renewable Energy Laboratory",
    year = "2017",
    title = "Route{E}: Route {E}nergy {P}rediction {M}odel",
    url = "https://www.nrel.gov/transportation/route-energy-prediction-model.html"
}

@misc{google_2023_environmental_report,
    author = "Google",
    organization = "Google Inc.",
    year = "2023",
    title = "Google 2023 {E}nvironmental {R}eport",
    url = "https://www.gstatic.com/gumdrop/sustainability/google-2023-environmental-report.pdf"
}

\end{filecontents}

\frenchspacing
\setlength{\pdfpagewidth}{8.5in}
\setlength{\pdfpageheight}{11in}
\pdfinfo{
/Title (Temperature-Dependent Estimation of Vehicle Fuel Economy)
/Author (Sindu Chitraju, Samuel Howard, Seif Ilkbarieh, Om Solanki, Andrew Wheeler)}
\setcounter{secnumdepth}{0}  
 \begin{document}

% The file aaai.sty is the style file for AAAI Press 
% proceedings, working notes, and technical reports.

% Remove the copyright from the footer
\nocopyright

\title{Temperature-Dependent Estimation of Vehicle Fuel Economy}
\author{Sindu Chitraju, Samuel Howard, Seif Ilkbarieh, Om Solanki, Andrew Wheeler\\
Tennessee Technological University
}

\maketitle

\begin{abstract}
\begin{quote}
    This project proposes to investigate the application of Recurrent Neural Networks (RNNs) to predict vehicle fuel economy using readily available on-board diagnostics II (OBD-II) data. 
    Leveraging the temporal dependencies inherent in driving patterns, we aim to develop a model that accurately predicts fuel consumption based on a sequence of sensor readings. 
    This approach offers a personalized and adaptable alternative to generalized fuel economy estimates, further contextualized with external climate data for even more accurate estimates.
    We will explore various RNN architectures and evaluate their performance using real-world driving data collected from a personal vehicle. 
    This research has the potential to contribute to fuel efficiency optimization strategies, personalized driver feedback systems, and more accurate route planning tools, building upon existing work like the Route Energy Prediction Model (REPM) used in Google Maps. 
\end{quote}
\end{abstract}

\section{Introduction}

\noindent Fuel economy is a critical concern with its economic and environmental implications. 
While standardized tests provide a general idea of the performance of a vehicle, real-world fuel consumption varies significantly based on driving style, traffic conditions, and other factors.
This project addresses the need for route-specific, weather dependent fuel economy prediction by leveraging the power of RNNs to model the complex temporal relationships within driving data.
We hypothesize that RNNs can learn the intricate dependencies between many of the variables that contribute to fuel economy including temperature, a metric that is not often accounted for.
The result of this work could empower drivers to optimize their driving habits whilst vehicle manufacturers could optimize aspects of vehicle drivetrains to account for environmental conditions.

\section{Related Work}

This project builds upon existing research in fuel economy modeling and route planning. 
The NREL's REPM, which powers fuel efficient routing in Google Maps, 
demonstrates the value of accurate fuel consumption prediction, 
preventing more than 1.2 million metric tons of carbon emissions \cite{google_2023_environmental_report}.
While REPM uses aggregated and generalized data from a few vehicles, 
our approach leverages the richness and breadth of vehicle sensor data from a single vehicle for better accuracy \cite{nrel_routee}.


\section{Methodology} % Drew
Our intended methodology for implementing the RNN-based machine learning model involves the creation of an end-to-end
framework. This toolkit will have the components required to pull the required data from the National Renewable
Energy Laboratory's (NREL) API, convert the sanitized dataset into the a form conducive to training a machine learning
model, and train the RNN model to predict fuel economy for a given trip.

Python offers the \emph{requests} module, which is designed to interface with HTTP-based APIs. This module will be at
the core of our data aggregation component as it provides all of the needed functionality in a commonly-used and
well-documented interface. This will be used in conjunction with the API provided by NREL to query vehicle and route
information. By doing this, we can build a dataset in preparation for training the RNN. In addition, we can query the
dataset arbitrarily to download new vehicle or route information as needed, allowing our dataset to grow dynamically
with our need.

Data sanitization will most likely have to be done manually, at least to some degree. Initial investigations of our
dataset indicate that it is relatively clean, and it provides data that is consistently formatted. However, should the
need arise for an individual to intervene and clean an anomalous piece of data, it will take place at this time.

Preprocessing of the constructed dataset will be required to present it in a form that is usable by a machine learning
model. During this time, data points irrelevant to our purposes can be omitted and the remaining information can be
encoded. The dataset can also be split into training and validation subsets at this point as well.

The final component will require training the Recurrent Neural Network to accurately predict vehicle fuel economy.
\emph{PyTorch} has been chosen as the machine learning framework for this project. It is a common library that
provides the means to both construct a RNN from scratch using built-in components and call upon a standardized class
that implements the architecture in its entirety. Having both of these options will provide some freedom for
experimenting with model structure and hyperparameters.

Once our model has been trained, we will be able to load a file containing the pre-trained weights. At this point,
further training can commence or the model can be used for inferencing previously-unseen data.



\section{Expected Results}

We expect that the model will outperform the RouteE model with real-world observed vehicular data. 
As our model will be working with environmental and route data, 
it will better be able to account for the conditions as well as the performance characteristics of the tested vehicle.
Furthermore, real-time inferencing using our model would allow a driver to glean yet greater fuel economy while driving.
Lastly, our model could enable a driver to save fuel by planning their trip ahead of time to optimize the time they leave for fuel economy.

\section{Conclusion and Future Work}

This project proposes to investigate the application of Recurrent Neural Networks (RNNs) to predict vehicle fuel economy using readily available on-board diagnostics II (OBD-II) data. 
By leveraging the temporal dependencies inherent in driving patterns, 
we aim to develop a personalized and adaptable model that accurately predicts fuel consumption based on a sequence of sensor readings. 
This approach offers a significant advantage over generalized fuel economy estimates, 
potentially empowering drivers to optimize their driving habits for improved fuel efficiency. 
Furthermore, we will investigate the potential of incorporating external climate data to enhance prediction accuracy over contemporary models.

The successful development of such a model has broad implications. 
It could contribute to personalized driver feedback systems, 
enabling real-time fuel efficiency coaching. 
Moreover, the predicted fuel economy could be integrated into route planning tools, 
complementing existing solutions like the Route Energy Prediction Model (REPM) used in Google Maps, 
by providing personalized and dynamic fuel consumption estimates whilest en route. 
This research aligns with the broader goals of promoting fuel conservation and sustainable transportation.

\bibliographystyle{aaai} 
\bibliography{references.bib}


\end{document}